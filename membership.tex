\section{Collaboration and Institution Membership}

\subsection{Requirements for Admission}

The CMB-S4 collaboration consists of Ph.D. scientists, engineers, Ph.D. thesis students, undergraduate students, and others who contribute significantly to the CMB-S4 program. A rough guideline is that all collaborators should devote a substantial fraction of their research time to the CMB-S4 program over a period of several years.
Visiting scientists and engineers (henceforth visitors), such as faculty members on sabbatical at a CMB- S4-affiliated institution, may also be admitted to the collaboration as short-term members, provided they devote a substantial portion of their research time to the collaboration.

The process of admission to the collaboration shall accord with the membership rules applicable to their degree of seniority. A senior member of the collaboration is a member from a collaborating institution who has a permanent appointment or an appointment that under normal circumstances can be expected to be renewed indefinitely. All other members of the collaboration are regular members. Senior members are admitted to the collaboration individually. Regular members are selected by their institutions. The only distinction between senior and regular members is the method by which they are admitted to the collaboration.
Senior members are admitted to the collaboration in one of four ways:
\begin{enumerate}
\item by being on the qualified elector list at the time of the initial spokesperson election,
\item by being on the approved list of senior members when their institution was admitted to the collaboration,
\item by being a regular member of the collaboration and receiving a position at any collaborating institution that would qualify him or her for senior membership status,
and
\item by individual application.
\end{enumerate}
A senior physicist at a collaborating institution who is not a member of the collaboration, and who wishes to join the collaboration, applies for membership in the same way as a new institution, as provided in Section 6. This procedure applies as well to visiting senior personnel applying for short-term membership.

The membership list is distinct from the author list described in Section \ref{sec:pub}.

\section{Admitting New Institutions}

To apply for membership in the CMB-S4 collaboration, representatives from the institution should first confer with the spokesperson. The representatives then submit a written proposal to the IB chair at least two weeks prior to the meeting of the IB at a regularly scheduled collaboration meeting. The IB chair circulates the proposal to IB members prior to the meeting and puts consideration of the application on the meeting agenda.

The proposal should describe the contributions that the prospective institution proposes to make to the collaboration and should include a list of the proposed senior members from the institution. The spokesperson will present the proposal to the collaboration by the prospective institution. This presentation should occur prior to the IB meeting held during the same collaboration meeting. The IB may: (1) approve the proposal and formally admit the institution to the collaboration at that meeting, (2) request further written clarifications to be considered at the next collaboration meeting, or (3) reject the proposal. The IB will attempt to make decisions on admission of new institutions by consensus. If consensus is not possible, a vote by secret ballot will be taken on admission. Admission will require a favorable vote by a super-majority of IB members present and voting.

\section{Removal of Collaborators and Institutions, Updating Membership and Leaving an Institution}

\subsection{Removal of Collaborators and Institutions}

Annually the IB Chair, in consultation with the spokesperson, will identify inactive institutions, using the criterion that an active institution has performed a reasonable amount of work on CMB-S4 within the previous twelve months. The IB may remove inactive institutions from the collaboration by a super- majority vote.

Individual collaborators who have not performed a reasonable amount of work on CMB-S4 for a period of at least twelve months, or whose behavior has been egregious and detrimental to CMB-S4, may be removed from the collaboration at an IB meeting by a super-majority vote of the IB members present and voting.

\subsection{Updating CMB-S4 Membership Rolls}

Each CMB-S4 IB member will be responsible for updating and verifying the collaboration membership list for his/her institution. A yearly questionnaire will be distributed by the IB chair to each institution to help keep track of all members and their status.

\subsection{Leaving a Collaborating Institution}

Individuals maintain their membership in the collaboration automatically for one year after they leave any collaborating institution. They can maintain a longer-term association with CMB-S4 by being ``adopted'' by a collaborating institution, with this process being approved by the IB on a case-by-case basis.

When a person's membership in the collaboration is terminated, they will no longer have access to the CMB-S4 document and database repository, internal forums, computing resources and data.


