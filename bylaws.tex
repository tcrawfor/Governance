\documentclass[12pt]{article}
\usepackage{amssymb,amsmath}
\usepackage[pdftex]{graphicx}
\usepackage{epsfig,subfigure}
\usepackage{epstopdf}
\usepackage{bm,url}
\DeclareGraphicsExtensions{.jpg,.pdf,.png,.eps,.ps}

\textheight = 522pt
\textwidth = 450pt
\oddsidemargin 0.0in


%% Define a new 'leo' style for the package that will use a smaller font.
\makeatletter
\def\url@leostyle{%
  \@ifundefined{selectfont}{\def\UrlFont{\sf}}{\def\UrlFont{\small\ttfamily}}}
\makeatother
%% Now actually use the newly defined style.
\urlstyle{leo}

\begin{document}

\title{CMB-S4 Collaboration Bylaws}
\maketitle

\tableofcontents

\section{Collaboration Governance Objectives/Preamble}

The CMB-S4 collaboration will carry out a CMB science program including the design and construction of the future experiment that is intended to be the definitive ground-based CMB program. CMB-S4 will deliver a highly constraining data set with which any model for the origin of the primordial fluctuations---be it inflation or an alternative theory---and their evolution to the structure seen in the Universe today must be consistent. This document outlines the CMB-S4 collaboration governance and organization of scientific activities.

\subsection{Voting Rules and Definitions}

Throughout this document, unless otherwise stated, a majority corresponds to more than one-half of the votes received, and a super-majority corresponds to more than two-thirds of the votes received. A vote of ``absent'' or an ``abstention'' does not count as a vote.

\section{Institutional Board (or Institutional Council)}

\subsection{Scope}

The Institutional Board (IB) controls the institutional aspects and the overall framework of the CMB-S4 collaboration. It is the ultimate policy forming body of the collaboration. It approves all membership issues, including the admission of new institutions into the collaboration, decides on the collaboration name and logo, sets the number of yearly collaboration meetings, organizes elections of the spokesperson and the Executive Board, develops a publications policy, appoints a publications board to approve physics results for publication, appoints a speakers board to select speakers for conferences and workshops, appoints an education and outreach board and has oversight and authority over the science policy and goals of the collaboration. The IB does not make technical decisions. The IB may change the bylaws at any time it feels it is necessary for the proper and efficient functioning of the collaboration (Section \ref{sec:amend}). The IB ratifies the Cooperative Agreements between the funding agencies and host laboratories/institutions. The IB may also establish standing and ad-hoc committees as necessary to perform its functions. Concerns of the Collaboration members are addressed to IB members who, when appropriate, bring those before the IB for its consideration. At the request of an IB member, the IB may require a detailed verbal, or written report from the Spokesperson on any action.

\subsection{Institutional Representation}

IB members will act as contacts for their institutions and inform the collaboration of any changes in personnel at their institutions. An IB member may designate another collaborator from the same institution to act as that institutions representative at any IB meeting. The CMB-S4 Spokesperson and Deputy, if they are not an IB representative, may attend all IB meetings as non-voting members except a special meeting convened to remove a Spokesperson (see Section \ref{sec:spokes}).

Each institution participating in CMB-S4 shall be represented by at most two IB members of whom one is voting whereas the other is a non-voting senior member. Each institution is responsible for choosing their own IB representative.

\subsection{IB Chair}

The IB chair is responsible for scheduling IB meetings, distributing the agendas, chairing the meetings, and distributing the minutes. The IB secretary can be delegated to take and distribute the minutes.

\subsection{Election of the IB Chair}

At the beginning of even numbered years the chair of the IB is elected from the IB membership to serve a two-year term. IB members nominate candidates by email and then vote by email, with the votes in both cases tabulated in secret by a third party. The candidate with the most votes becomes the IB chair. In case of a tie, a runoff election is held. In case of a tie in the final runoff, the chair will recuse him/herself from the vote to resolve the tie, but otherwise he/she is eligible to vote in the election. The secretary of the IB, should one be used, is chosen informally by the chairman of the IB to serve during his or her term.

\subsection{IB Meetings}

The IB meets in closed session at every CMB-S4 collaboration meeting and in extraordinary meetings held by telephone or video conference on the initiative of the IB chair or at the request of three IB representatives. IB votes require a majority of those present, unless otherwise specified in the bylaws. IB votes can be held by email, unless otherwise specified in the bylaws. Minutes of the IB meetings are distributed to IB members by the IB chair or IB secretary.

\section{Spokesperson/Deputy or Co-Spokespersons}
\label{sec:spokes}

\subsection{Scope}

The	scientific	leadership	of	the	CMB-S4	collaboration	resides	with the	spokesperson	(or	consists	of	two	
equal	co-spokespersons).			The	spokesperson	participates	actively	in	the	management	of	all	aspects	of	
the	collaboration.	They	are	expected	to	solicit	advice	from	the	collaboration	at	large	and	from	the	
executive	board	on	physics,	technical,	management,	and	leadership	issues.	The	spokesperson	is
responsible	for	the	creation	and	modification	of	management	structures	and	for	appointments	to	
leadership	positions	in	physics,	technical,	and	management	areas.	They	serve	as	the	primary	contacts	
with	a	future	host	laboratory/institution,	the	funding	agencies,	scientific	organizations,	and	the	press.
They	also	organize	and	run	the	collaboration	meetings.	The	spokesperson	may	establish	standing	and	
ad-hoc	committees	as	necessary.

\subsection{Election of a Spokesperson}

Spokespersons are elected to two-year terms. Normal spokesperson elections will be timed such that a slate of nominees can be presented at the IB meeting closest to January 1 each year. Prior to the election, qualified electors (see Article 9) are asked to nominate individuals by the IB chair. Each elector may nominate one candidate. A minimum of two nominations are required for a nominee to be eligible to appear on the slate of election candidates. After a nomination period, the IB chair will consult with the nominees to ascertain their willingness to stand for the election.

If no candidate agrees to stand for election, the terminating spokesperson will continue to serve for six months, after which a special election will be held. The term of the individual elected in a special election will be reduced by the amount needed to cause the sum of his/her term and the additional months served by the terminating spokesperson to equal two years.

The election is held by email, with votes submitted to the IB chair by qualified electors. Each candidate will provide a Candidate Statement and a short CV to the IB chair, who will collect these documents and make them available to the collaboration via the CMB-S4 internal web page prior to the balloting. The IB chair determines the detailed timing of the balloting and determines procedures for dealing with any election situations not anticipated in the bylaws. Each qualified elector votes for a single candidate for each vacant position.

The IB chair checks the ballots against the list of qualified voters and tallies the votes. If no candidate receives a majority of the votes cast, a runoff election will be held between the two candidates with the most votes. The elected candidate will take office when the IB chair announces the results of the election to the collaboration by email. In the event that the IB chair is one of the candidates, then the election will be run as detailed in this section by the IB secretary, or, if they are also a candidate, by the last elected spokesperson.

\subsection{Selection of a Deputy Spokesperson}

The Spokesperson is responsible for selecting a Deputy, with the IB concurrence, who will oversee many of the ongoing operations of the Collaboration under the direction of the Spokesperson. The Deputy will work close with and act as assistant to the Spokesperson. The Deputy must be a regular member.

\subsection{Removal of a Spokesperson}

The IB can remove a spokesperson by a secret ballot, requiring approval by at least supermajority of those present and voting at a regularly scheduled IB meeting. A two-week notice is required for a vote for removal and such a vote will be taken if at least half the IB members support such a proposal to the chair. In the event of a removal, the open position(s) will be filled following the procedures in these bylaws.

\section{Executive Board}

\subsection{Scope}

The Executive Board (EB) advises the spokesperson on scientific, financial, and organizational choices and policies. The EB also establishes procedures for making technical choices (e.g., creating technical panels) and for operating the future experiment. The EB may establish standing and ad-hoc committees as necessary to perform its functions. The EB will meet at each collaboration meeting and when requested by the spokesperson or at least two members of the EB. The deputy spokesperson is responsible for producing minutes or summaries of EB meetings, which will be made available to the CMB-S4 membership.

\subsection{EB Membership}

The EB consists of six elected members and three ex-officio members: the spokesperson and deputy spokesperson and the IB chair. The spokesperson serves as the EB chair.

\subsection{Election of EB Members}

Three members of the EB are chosen each year for a two-year term, at the IB meeting closest to January 1. Qualified electors (see Section 2) submit nominations to the IB chair prior to the election. Each collaborator may nominate one candidate per vacant position. After a nomination period, the IB chair will consult with the nominees to ascertain their willingness to stand for election. Candidates with the most nominations are listed on the ballot, until the number of candidates equals twice the number of EB vacancies. The number of candidates on the ballot may exceed twice the number of vacancies in the event that several candidates are tied with the minimum number of nominations.

The election is held by email, submitted to the IB chair by qualified electors. The IB chair determines the detailed timing of the balloting. Each qualified elector votes for as many candidates as there are vacant positions. The IB chair checks the ballots against the list of qualified electors and tallies the votes. The candidates with the highest number of votes are elected to fill all vacant positions. A runoff election is held in the event of a tie. The elected candidates will take office when the IB chair announces the results of the election to the collaboration by email.

In the first election six EB members will be elected; the three receiving the largest number of votes will serve terms of no less than three years, the others will serve terms of no less than two years. Thereafter each EB member will serve two-year terms, with three being up for election every year.

If a member of the EB resigns or becomes unable to fulfill his duties, there will be a special election to fill the vacancy. If a newly elected spokesperson is a current Executive Board member, an election for a new EB member will be held immediately following the election of the spokesperson for the remaining EB term of the new spokesperson.

The IB chair will determine procedures for dealing with any election situations not anticipated in these bylaws.

The IB may remove any EB member by the same procedures described in Article 6 for removal of a spokesperson.

\section{Collaboration and Institution Membership}

\subsection{Requirements for Admission}

The CMB-S4 collaboration consists of Ph.D. scientists, engineers, Ph.D. thesis students, undergraduate students, and others who contribute significantly to the CMB-S4 program. A rough guideline is that all collaborators should devote a substantial fraction of their research time to the CMB-S4 program over a period of several years.
Visiting scientists and engineers (henceforth visitors), such as faculty members on sabbatical at a CMB- S4-affiliated institution, may also be admitted to the collaboration as short-term members, provided they devote a substantial portion of their research time to the collaboration.

The process of admission to the collaboration shall accord with the membership rules applicable to their degree of seniority. A senior member of the collaboration is a member from a collaborating institution who has a permanent appointment or an appointment that under normal circumstances can be expected to be renewed indefinitely. All other members of the collaboration are regular members. Senior members are admitted to the collaboration individually. Regular members are selected by their institutions. The only distinction between senior and regular members is the method by which they are admitted to the collaboration.
Senior members are admitted to the collaboration in one of four ways:
\begin{enumerate}
\item by being on the qualified elector list at the time of the initial spokesperson election,
\item by being on the approved list of senior members when their institution was admitted to the collaboration,
\item by being a regular member of the collaboration and receiving a position at any collaborating institution that would qualify him or her for senior membership status,
and
\item by individual application.
\end{enumerate}
A senior physicist at a collaborating institution who is not a member of the collaboration, and who wishes to join the collaboration, applies for membership in the same way as a new institution, as provided in Section 6. This procedure applies as well to visiting senior personnel applying for short-term membership.

The membership list is distinct from the author list described in Section \ref{sec:pub}.

\section{Admitting New Institutions}

To apply for membership in the CMB-S4 collaboration, representatives from the institution should first confer with the spokesperson. The representatives then submit a written proposal to the IB chair at least two weeks prior to the meeting of the IB at a regularly scheduled collaboration meeting. The IB chair circulates the proposal to IB members prior to the meeting and puts consideration of the application on the meeting agenda.

The proposal should describe the contributions that the prospective institution proposes to make to the collaboration and should include a list of the proposed senior members from the institution. The spokesperson will present the proposal to the collaboration by the prospective institution. This presentation should occur prior to the IB meeting held during the same collaboration meeting. The IB may: (1) approve the proposal and formally admit the institution to the collaboration at that meeting, (2) request further written clarifications to be considered at the next collaboration meeting, or (3) reject the proposal. The IB will attempt to make decisions on admission of new institutions by consensus. If consensus is not possible, a vote by secret ballot will be taken on admission. Admission will require a favorable vote by a super-majority of IB members present and voting.

\section{Removal of Collaborators and Institutions, Updating Membership and Leaving an Institution}

\subsection{Removal of Collaborators and Institutions}

Annually the IB Chair, in consultation with the spokesperson, will identify inactive institutions, using the criterion that an active institution has performed a reasonable amount of work on CMB-S4 within the previous twelve months. The IB may remove inactive institutions from the collaboration by a super- majority vote.

Individual collaborators who have not performed a reasonable amount of work on CMB-S4 for a period of at least twelve months, or whose behavior has been egregious and detrimental to CMB-S4, may be removed from the collaboration at an IB meeting by a super-majority vote of the IB members present and voting.

\subsection{Updating CMB-S4 Membership Rolls}

Each CMB-S4 IB member will be responsible for updating and verifying the collaboration membership list for his/her institution. A yearly questionnaire will be distributed by the IB chair to each institution to help keep track of all members and their status.

\subsection{Leaving a Collaborating Institution}

Individuals maintain their membership in the collaboration automatically for one year after they leave any collaborating institution. They can maintain a longer-term association with CMB-S4 by being ``adopted'' by a collaborating institution, with this process being approved by the IB on a case-by-case basis.

When a person's membership in the collaboration is terminated, they will no longer have access to the CMB-S4 document and database repository, internal forums, computing resources and data.




\section{Education and Outreach}

The CMB-S4 collaboration collectively and individually participates in and provides support for efforts in public outreach and education on subjects related to its science. The Spokesperson, with the IB concurrence, responds to requests for information from the media or may take the initiative providing material. The Spokesperson, with the IB concurrence and utilizing the advice from the Education and Outreach committee, appoints a Collaboration member to lead an education program for students and teachers at all levels. The Collaboration maintains coordination and cooperation with other ongoing education initiatives. All new scientific material to be released for purposes of public outreach or education containing other than previously published data or results must have been agreed upon by the IB and approved through the Education and Outreach Committee (Section \ref{sec:eando}).

\section{CMB-S4 Authorship Policy}
\label{sec:pub}

\subsection{Scope}
The authorship policy spells out the criteria and procedures for determining which members of the collaboration will be listed as authors on publications. The author list will be comprised primarily of currently active members of the collaboration that satisfy the effort and shift criteria discussed below. Former members of the collaboration that have recently left can also opt to be included as legacy authors. Procedures for adding special cases and for addressing appeals are also specified. The formulation of the author list is treated separately for physics papers and technical papers. In general, the author list for a given paper is considered frozen once the paper has been submitted. Exceptions to this are discussed in Section~\ref{sec:except}.


\subsection{Current Authors}
Current authors are scientists at the level of graduate student or above who are members of the CMB- S4 collaboration and are actively engaged in the collaboration. This includes visitors. They will have contributed a significant fraction of their research time to the collaboration in the twelve months prior to preparation of the author list.

The initial set of proposed authors will be submitted to the IB by each of the IB representatives, who will prepare a list of proposed names at their institution, along with a brief description of their contributions and level-of-effort. The IB chair will circulate this initial list to the IB one month in advance of the vote. IB members who want a specific candidate discussed will submit the name of the candidate to the IB chair no less than two weeks before the vote, with any questions or issues, and that name will be flagged for discussion.

The IB chair will send the names, questions and issues regarding these candidates to all IB representatives promptly. At the initial vote, candidates who have not been flagged can be accepted by acclamation; flagged names will be discussed individually with the IB representative making the case for the candidate. A majority vote of voting members will suffice to include the candidate. This group will form the default author list for all qualifying CMB-S4 publications as defined in Section~\ref{sec:pubdef}. Individual authors can opt out of being included as an author for a given publication by notifying the spokesperson.

The current author list will be updated through this mechanism as appropriate but at least once per year, with the IB chair supervising the update. For the updates, the IB representatives will flag any new additions to the author list. Once formed, the current author list will be made available to the collaboration. Any individual member of the collaboration can petition the chair of the IB for inclusion if they believe to have been improperly excluded.

\subsection{Voting Rules}

A majority vote of those voting is required, and all authorship votes follow this rule unless otherwise noted.

\subsection{Definition of Qualifying Publications}

The current author list shall be attached to any publication, as defined in Section~\ref{sec:pubdef}, discussing a physics result, as defined in Section~\ref{sec:pubdef}. In addition, when allowed, it shall also be attached to conference proceedings.

\subsection{Legacy Authors}
Legacy authors are individuals who have ceased active participation in the collaboration preceding submission of a publication but choose to opt-in to the author list via email to the spokesperson. The opt-in can either be a blanket opt-in, whereby the individual?s name is automatically included in the author list for all papers, or a case-by-case opt-in, whereby the individual must contact the spokesperson for each submitted paper. An individual can exercise his or her opt-in option until their legacy period has expired. For deceased members, the blanket opt-in option will be assumed. The legacy period, which applies to visitors as well, will be set to either one year or one-quarter of the time since the person has joined the collaboration, whichever is longer. The time a person joined shall be taken from the collaboration list as maintained by the IB chair. Once the legacy period has expired the Special Author rules apply.

\subsection{Special Authors}
Special authors are defined as authors who do not qualify as current or legacy authors but are listed on an individual paper based on specific contributions to that paper. Engineering staff, technicians, undergraduates, and other non-physicists who have made exceptional contributions are examples of special authors, as well as scientists who have made significant contributions to the specific paper. They must be nominated for inclusion on the author list by a member of the IB.

Visitors, whose contact with the collaboration may encompass a period of less than one year, may be accepted as authors on papers concerning topics to which they have made significant contributions, subject to the legacy period and the approval of the IB.

For the first paper reporting, CMB-S4 individuals can either be nominated by a member of the IB, or can petition the IB chair for inclusion on the author list. In either case, both of the following conditions must be met: (1) the individual must have worked on the collaboration for at least one FTE year, and (2) the individual must have made an intellectual contribution to the collaboration. The IB must approve special author nominations and petitions by a majority vote.

\subsection{Technical Publication Authors}

Publications that discuss technical results, as defined in Section~\ref{sec:pubdef}, will not be required to carry the current author list. Publications in NIM, IEEE, SPIE on hardware, or specialized simulations are typical examples. The primary proponents of technical papers will create a custom author list that does not require approval of the collaboration. However, the draft manuscript must be submitted to the Publications Board (cf. Section~\ref{sec:pubdef}) and the author list to the IB. IB representatives can ask the proponents to add individuals they think ought be included on the author list.

\subsection{Examples of Contributions Qualifying for Authorship}

The following list gives examples of contributions; it is meant to be indicative, not exclusive, and to serve as a guideline:
\begin{itemize}
\item governance of the collaboration
\item construction of the experiment
\item maintenance or operation of the experiment ? scientific effort during the project phase
\item simulation or data reconstruction work
\end{itemize}

\section{Amendments to Bylaws}
\label{sec:amend}

These bylaws may be amended by the IB. A new bylaw or an amendment may be presented to the IB at one of its regularly scheduled meetings for consideration. Amendments may be introduced by any IB member or by a spokesperson. Proposed amendments will be distributed by email to the entire collaboration for their comments following this meeting. The IB will then consider the amendment for adoption at the next scheduled IB meeting. Approval of an amendment requires a super-majority vote of the IB members present.


\end{document}
